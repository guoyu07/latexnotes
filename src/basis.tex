\part{Basis}



\chapter{Example}



\begin{lstlisting}[language=TeX]
% !TEX TS-program = xelatex
% !TEX encoding = UTF-8

% This is a simple example for a XeLaTeX document using the
% "article" class, with the fontspec package to easily select fonts.
\document{article}

\title{Example}
\author{theqiong.com}

\begin{document}

\maketitle
\tableofcontents
\section{Introduction}
\section{Principle}

\bibliography{example}

\end{document}
\end{lstlisting}



\section{Comment}


注释以\%开头。

\begin{compactitem}
\item 第一行注释表明文档编码(例如UTF8);
\item 第二行注释说明文档名称;
\item 第三行注释说明源文件内容。
\end{compactitem}


\section{Encoding}

一般情况下使用UTF8编码(尤其对于中文文档)。




\section{Class}




对于中文文档使用的ctex宏包,可以将文档类分为ctexart、article和book等。



\section{Title}

在title页可以声明整个文档的标题、作者和写作日期。

\begin{compactitem}
\item \texttt{\textbackslash title\{\}}:标题
\item \texttt{\textbackslash author\{\}}:作者
\item \texttt{\textbackslash date\{\}}:日期
\item \texttt{\textbackslash today}:当前日期
\end{compactitem}


在文档主体中,使用\texttt{\textbackslash maketitle}来生成标题页。

\section{Preamble}


在\texttt{\textbackslash begin\{document\}}之前的部分称为导言区(Preamble),可以对文档的性质进行设置,以及设置自定义命令。


\section{Abstract}

文档的摘要在\texttt{\textbackslash maketitle}之后使用\texttt{abstract}环境来生成。

\begin{lstlisting}[language=TeX]
\begin{abstract}
这是一篇关于LaTeX的笔记。
\end{abstract}
\end{lstlisting}


\section{Body}


文档的主体部分也就是在\texttt{\textbackslash begin\{document\}}和\texttt{\textbackslash end\{document\}}之间声明的document环境中的部分,也是直接输出的部分。

\section{Catalog}

在生成标题页的命令之后的\texttt{\textbackslash tableofcontents}可以用来生成目录页,类似的命令还有\texttt{\textbackslash listoffigures}以及\texttt{\textbackslash listoftables}等。


\section{Index}


在导言区的\texttt{\textbackslash title\{\}}之前执行\texttt{\textbackslash makeindex}命令之后,在生成目录页的命令后面再通过\texttt{\textbackslash printindex}可以为文档生成索引,不过要配合hyperref宏包才能显示。



\section{Bibliography}

在文档主体部分通过\texttt{\textbackslash bibliography\{\}}命令来引入参考文献,并且使用\texttt{\textbackslash bibliographystyle}设置样式。


\begin{lstlisting}[language=TeX]
xelatex test.tex
bibtex test.tex
xelatex test.tex
xelatex test.tex
\end{lstlisting}


相比而言,最简单的自动化工具是页码、定理和公式的自动编号,其余的包括生成目录和图表公式的交叉引用,最复杂的自动化工具包括文献自动引用等。

\section{Quotation}

在正文中,使用\texttt{quote}环境可以输入引用信息,并且将环境中的内容单独成行,增加缩进和上下间距来突出引用。

默认情况下,\texttt{quote}环境的不足在于无法改变内容的字体,需要用户自己设置字体等。


\section{Footnote}

\texttt{\textbackslash footnote\{\}}命令可以生成脚注。


\section{Declaration}

\textbackslash{\textbackslash zihao}和\texttt{\textbackslash kaishu}等命令会影响后面的问题,直到整个分组结束,需要使用大括号限制作用范围,因此这类名称称为声明(declaration)。

\chapter{Environment}


分组限定了声明的作用范围,一个\LaTeX 环境自然就是一个分组(roup),最大的分组是表示正文的\texttt{document}环境。

\LaTeX 环境(environment)的一般格式如下,而且环境也可以设置参数。


\begin{lstlisting}[language=TeX]
\begin{<环境名>}[<可选参数>]<其他参数>
<环境内容>
\end{<环境名>}
\end{lstlisting}

例如,定理就是使用\texttt{theorem}产生的,而且用户可以在导言区自己定义定理环境。

\begin{lstlisting}[language=TeX]
\newtheorem{thm}{定理}
\begin{thm}[勾股定理]
直角三角形斜边的平方等于两腰的平方和。
\end{thm}
\end{lstlisting}

\begin{thm}[勾股定理]
直角三角形斜边的平方等于两腰的平方和。
\end{thm}



\texttt{\textbackslash environment}命令可以用来定义新的环境,例如可以在原来\texttt{quote}的基础上再增加格式控制。


\begin{lstlisting}[language=TeX]
\newenvironment{myquote}
	{\begin{quote}\kaishu\zihao{-5}}
	{\end{quote}}
\end{lstlisting}

除了使用环境,使用成对的大括号也可以产生一个分组,而且字号、字体等命令可以影响整个环境。



\chapter{Formula}

数学公式可以分为行内公式和列表公式。

\begin{compactitem}
\item 行文中的公式称为“正文公式”(in-text formula)或“行内公式”(inline formula),可以使用\texttt{\$\$}进行输入。
\item 单独居中的公式称为“显示公式”或“列表公式”(displayed formula),可以使用\texttt{equation}环境进行输入。
\end{compactitem}

\texttt{amsmath}宏包等提供了数学符号的\LaTeX 输入,例如“角”($\angle$)的符号是\texttt{\textbackslash angle},字母$\pi$\footnote{ISO标准对科技文档要求常数$\uppi$使用直立体。}可以使用\texttt{\$\textbackslash pi\$}表示。

在数学公式中可以输入上下标、分式和根式等。

\begin{compactitem}
\item \texttt{\^{}}表示引入一个上标;
\item \texttt{\_}表示引入一个下标。
\end{compactitem}


\chapter{Figure}

在\LaTeX 中使用插图有两种方式,可以插入事先准备好的图片,或者使用\LaTeX 直接在文档中绘图。

插图功能不是由\LaTeX 的内核内置的,一般使用\texttt{graphicx}来提供插图环境。

\begin{lstlisting}[language=TeX]
\usepackage{graphicx}
\end{lstlisting}

在引入\texttt{graphicx}宏包后就可以使用\texttt{\textbackslash includegraphics\{\}}\footnote{除了一些很小的ICON图形之外,很少将插入直接插入行内文字中,一般都是使用单独的图形环境\texttt{figure}来插入图形。}来插入图片,而且可以通过\texttt{width}、\texttt{height}和\texttt{scale}等参数来对图片尺寸进行调整。

\LaTeX 支持的图形文件格式与所使用的编译程序有关,例如\XeLaTeX 支持的图形文件格式包括pdf、png、jpg和eps等。

在\LaTeX 文档中插入的图形就是一个有内容的矩形盒子,在正文中和一个很大的字符没有区别。

通常情况下,图形会被放在一个可以改变相对位置的环境中变成浮动体(float),而且在浮动体中还可以给图形加入说明性的标题(\texttt{\textbackslash caption\{\}})和标签(\texttt{\textbackslash label\{\}})。

\begin{compactitem}
\item h表示浮动体可以出现在环境周围的文本所在处或当前位置(here);
\item t表示浮动体可以出现一页的顶部(top);
\item b表示浮动体可以出现在一页的底部(bottom);
\item p表示浮动体可以出现在只允许出现浮动对象的页面上。
\end{compactitem}

\texttt{figure}环境内部相当于普通的段落(默认没有缩进),\texttt{\textbackslash centering}表示后面的内容居中。


\texttt{caption}宏包可以用来设置图表标题格式。例如,下面的设置将设定所有图表标题使用悬挂对齐方式(即编号向左突出),整体使用小号字,标题文本使用斜体(对汉字来说就是楷书)。


\begin{lstlisting}[language=TeX]
\usepackage[format=hang, font=small, textfont=it]{caption}
\end{lstlisting}



\chapter{Table}


表格一般都是在\LaTeX 内部使用代码输出的,其中需要确定表格的行、列对齐模式和表格线。

表格和\texttt{\textbackslash includegraphics}命令插入的图形一样,都是一个大型盒子,而且都是放在浮动环境中。


\begin{compactitem}
\item 插图使用\texttt{figure}环境;
\item 表示使用\texttt{table}环境。
\end{compactitem}


如果在\texttt{figure}或\texttt{table}环境的参数中使用[H],则表示“就放在这里,不浮动”。

注意,[H]选项并不是标准\LaTeX 的\texttt{table}环境使用的参数,而是由\texttt{float}宏包提供的特殊功能。


\begin{lstlisting}[language=TeX]
\usepackage{float}
\end{lstlisting}


\chapter{Dimension}

\begin{compactitem}
\item \texttt{\textbackslash qquad}可以产生长度为2em的空白;
\item \texttt{\textbackslash quad}可以产生长度为em的空白;
\end{compactitem}

\chapter{Reference}

引用不局限于参考文献,只要事先设定了标签,图表和公式的编号也可以通过辅助文件为中介进行引用。

最基本的交叉引用命令是\texttt{\textbackslash ref},它可以以标签为参数,得到被引用的编号。

另外,在数学宏包\texttt{amsmath}中还定义了\texttt{\textbackslash eqref}命令来专门处理公式的引用,并能够产生括号。

\chapter{Geometry}

\texttt{geometry}可以用来设置页面尺寸。



\begin{lstlisting}[language=TeX]
\usepackage{geometry}
\geometry{a4paper, centering, scale=0.8}
\end{lstlisting}




\begin{lstlisting}[language=TeX]

\end{lstlisting}




\begin{lstlisting}[language=TeX]

\end{lstlisting}



\begin{lstlisting}[language=TeX]

\end{lstlisting}



\begin{lstlisting}[language=TeX]

\end{lstlisting}


\begin{lstlisting}[language=TeX]

\end{lstlisting}


\begin{lstlisting}[language=TeX]

\end{lstlisting}



\begin{lstlisting}[language=TeX]

\end{lstlisting}



\begin{lstlisting}[language=TeX]

\end{lstlisting}



\begin{lstlisting}[language=TeX]

\end{lstlisting}



\begin{lstlisting}[language=TeX]

\end{lstlisting}



\begin{lstlisting}[language=TeX]

\end{lstlisting}



\begin{lstlisting}[language=TeX]

\end{lstlisting}




\begin{lstlisting}[language=TeX]

\end{lstlisting}



\begin{lstlisting}[language=TeX]

\end{lstlisting}



\begin{lstlisting}[language=TeX]

\end{lstlisting}




\begin{lstlisting}[language=TeX]

\end{lstlisting}




\begin{lstlisting}[language=TeX]

\end{lstlisting}



\begin{lstlisting}[language=TeX]

\end{lstlisting}



\begin{lstlisting}[language=TeX]

\end{lstlisting}


\begin{lstlisting}[language=TeX]

\end{lstlisting}


\begin{lstlisting}[language=TeX]

\end{lstlisting}



\begin{lstlisting}[language=TeX]

\end{lstlisting}



\begin{lstlisting}[language=TeX]

\end{lstlisting}



\begin{lstlisting}[language=TeX]

\end{lstlisting}



\begin{lstlisting}[language=TeX]

\end{lstlisting}



\begin{lstlisting}[language=TeX]

\end{lstlisting}



\begin{lstlisting}[language=TeX]

\end{lstlisting}



\begin{lstlisting}[language=TeX]

\end{lstlisting}



\begin{lstlisting}[language=TeX]

\end{lstlisting}



\begin{lstlisting}[language=TeX]

\end{lstlisting}




\begin{lstlisting}[language=TeX]

\end{lstlisting}




\begin{lstlisting}[language=TeX]

\end{lstlisting}



\begin{lstlisting}[language=TeX]

\end{lstlisting}



\begin{lstlisting}[language=TeX]

\end{lstlisting}


\begin{lstlisting}[language=TeX]

\end{lstlisting}


\begin{lstlisting}[language=TeX]

\end{lstlisting}



\begin{lstlisting}[language=TeX]

\end{lstlisting}



\begin{lstlisting}[language=TeX]

\end{lstlisting}



\begin{lstlisting}[language=TeX]

\end{lstlisting}



\begin{lstlisting}[language=TeX]

\end{lstlisting}



\begin{lstlisting}[language=TeX]

\end{lstlisting}



\begin{lstlisting}[language=TeX]

\end{lstlisting}




\begin{lstlisting}[language=TeX]

\end{lstlisting}



\begin{lstlisting}[language=TeX]

\end{lstlisting}



\begin{lstlisting}[language=TeX]

\end{lstlisting}




\begin{lstlisting}[language=TeX]

\end{lstlisting}




\begin{lstlisting}[language=TeX]

\end{lstlisting}



\begin{lstlisting}[language=TeX]

\end{lstlisting}



\begin{lstlisting}[language=TeX]

\end{lstlisting}


\begin{lstlisting}[language=TeX]

\end{lstlisting}


\begin{lstlisting}[language=TeX]

\end{lstlisting}



\begin{lstlisting}[language=TeX]

\end{lstlisting}



\begin{lstlisting}[language=TeX]

\end{lstlisting}



\begin{lstlisting}[language=TeX]

\end{lstlisting}



\begin{lstlisting}[language=TeX]

\end{lstlisting}



\begin{lstlisting}[language=TeX]

\end{lstlisting}



\begin{lstlisting}[language=TeX]

\end{lstlisting}




\begin{lstlisting}[language=TeX]

\end{lstlisting}



\begin{lstlisting}[language=TeX]

\end{lstlisting}



\begin{lstlisting}[language=TeX]

\end{lstlisting}




\begin{lstlisting}[language=TeX]

\end{lstlisting}




\begin{lstlisting}[language=TeX]

\end{lstlisting}



\begin{lstlisting}[language=TeX]

\end{lstlisting}



\begin{lstlisting}[language=TeX]

\end{lstlisting}


\begin{lstlisting}[language=TeX]

\end{lstlisting}


\begin{lstlisting}[language=TeX]

\end{lstlisting}



\begin{lstlisting}[language=TeX]

\end{lstlisting}



\begin{lstlisting}[language=TeX]

\end{lstlisting}



\begin{lstlisting}[language=TeX]

\end{lstlisting}



\begin{lstlisting}[language=TeX]

\end{lstlisting}



\begin{lstlisting}[language=TeX]

\end{lstlisting}



\begin{lstlisting}[language=TeX]

\end{lstlisting}